\chapter{Future Work}
\label{chapter:future_work}

  Several issues need to be resolved before MMC is ready to be tested in a
  production setting. Furthermore, deeper research into particular aspects of
  the MMC algorithm could yield improved performance. In this chapter I outline
  many of these important questions.

\begin{enumerate}

\item
  Can we identify the page with the smallest expected value, or close to the
  smallest expected value, in less than $O(|K|)$ time?

\item
  Can a mixture model be used to provide a caching algorithm that is local to a
  process or a thread, but where the amount of main memory that a process can
  utilize is governed by a mixing parameter?

\item
  If we can estimate the expected headway between faults, can we use this
  information to inform the scheduler?

\item
  Is there any benefit to looking at the relationship between page locations? For
  example, a page hit occurs when the page is re-referenced while it is in
  cache. However, what if we count a reference to page $x + 1$ as a partial
  hit to page $x$?

\item
  Can we use more flexible distribution families to describe the shape of the
  source distributions?

\item
  What other measurements can we take for page requests, and do these
  measurements help improve the hit rate of the MMC algorithm?

\item
  Do alternatives to the EM algorithm exist that require drastically less
  processing time?

\item
  Can a probabilistic model for prefetching be described that will fit into the
  MMC mixture model?

\item
  What design modifications need to be made to use fixed-point arithmetic rather
  than floating-point arithmetic for all of the internal calculations?

Addressing these questions will allow us to take full advantage of the strengths
of the MMC algorithm. Virtual memory is one of the fundamental areas of
systems research, but caches have traditionally been a black box. Development
into statistical cache analysis techniques and algorithms could yield vital
insights and enhancements.

\end{enumerate}

